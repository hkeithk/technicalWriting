\documentclass{beamer}

\mode<presentation>
{
  \usetheme{default}      % or try Darmstadt, Madrid, Warsaw,
  \usecolortheme{default} % or try albatross, beaver, crane, ...
  \usefonttheme{default}  % or try serif, structurebold, ...
  \setbeamertemplate{navigation symbols}{}
  \setbeamertemplate{caption}[numbered]
} 

\usepackage[natbib=true,style=numeric,backend=bibtex,useprefix=true]{biblatex}
\addbibresource{bib.bib}

\usepackage[english]{babel}
\usepackage[utf8x]{inputenc}

\usepackage{biblatex}
\bibliography{bib.bib}

\title[Your Short Title]{Data Mining for Strategic Product Prioritization}
\author{Keith Ho}
\institute{University of California, Santa Cruz}
\date{November 28th, 2018}

\begin{document}

\begin{frame}
  \titlepage
\end{frame}

\begin{frame}{Outline}
  \tableofcontents
\end{frame}

\section{Introduction}

\begin{frame}{Introduction}

\begin{itemize}
  \item What is Voice of the Customer? 
  \item How do we currently manage customer data?
\end{itemize}

\vskip 1cm
\end{frame}

\section{The Hybrid Analytical Framework}
\begin{frame}{The Hybrid Analytical Framework}
Steps:
\begin{columns}
\column{0.5\textwidth}
    \begin{enumerate}
        \item Process raw text data
        \item Categorize keyword clusters
        \item Sum importance of clusters
        \item Apply SELRank for prioritization
     \end{enumerate}

\column{0.5\textwidth}
    \begin{figure}
    \includegraphics[width=\textwidth]{images/Figure1.png}
    %\caption{\label{fig:1}}
    \end{figure}
\end{columns}
\end{frame}
%--------------end of page 4
\section{Discovering Feature Clusters using the Knowledge Transformation Model}
\begin{frame}{Discovering Feature Clusters using the Knowledge Transformation Model}
X $\approx FSG^T$

\begin{itemize}
    \item X = Word Request semantic Matrix
    \item F = Nonnegative matrix representing prior knowledge in the word space
    \item G = a matrix representing knowledge in the document space
    \item S = a matrix providing a condensed view of X
\end{itemize}
    
$$\underset{F,G,S}{min} || X- FSG^T||^2+\alpha||F-F_0||^2.$$
    
\begin{itemize}
    \item After the feature clusters are obtained, we can know what major features customers are asking for
\end{itemize}
\end{frame}
%----end of page 5
\section{Prioritizing Customer Requests}
\begin{frame}{Prioritizing Customer Requests}
\begin{columns}
    \column{0.5\textwidth}
        Semantic Enhanced Link-based Ranking Algorithm
        \includegraphics[width=\textwidth]{images/Algorithm.png}
        
    \column{0.5\textwidth}
    Link Structure
    \includegraphics[width=\textwidth]{images/Figure2.png}
\end{columns}
\end{frame}
%---- end of page 6

\section{SELRank vs. Pagerank}
\begin{frame}{SELRank vs. Pagerank}
\includegraphics[width=\textwidth]{images/Figure4.png}
\end{frame}
%-----end of page 7

%----end of page 8
\section{Other Web-Based Ranking Algorithms}
\begin{frame}{Other Web-Based Ranking Algorithms}

\begin{columns}
\column{0.3\textwidth}
    \begin{enumerate}
        \item HITS
        \item Weighted PageRank
        \item Topic Sensitive Pagerank
        \item SELRank
    \end{enumerate}
    
\column{0.7\textwidth}
    \includegraphics[width=\textwidth]{images/Table1.png}
\end{columns}

\end{frame}

%-----end of page 9

\section{Case study}
\begin{frame}{Case Study}
\includegraphics[width=0.5\textwidth]{images/Table3.png}
\includegraphics[height=5cm, width=0.5\textwidth]{images/Figure5.png}
\end{frame}

%-----end of page 10
\section{Conclusion}
\begin{frame}{Conclusion}
\begin{center}
    \begin{itemize}
        \item We were successful in applying the framework on Xerox's FER data
    \end{itemize}
    \includegraphics[width=\textwidth]{images/Figure7.png}
\end{center}
\end{frame}
%----end of page 11

\section{References}
\begin{frame}{References}
\bibliographystyle{apalike}

    \begin{thebibliography}{10}
    \beamertemplatearticlebibitems
    \bibitem{haveli}
    Haveli, T.H. 2002. Topic-sensitive pagerank: \emph{In Proceedings of the International Conference on World Wide Web (WWW'02) .}517-526.
    
    \beamertemplatearticlebibitems
    \bibitem{Wu} Wu, J. and Aberer, K. 2005. Using a layered markov model for distributed web ranking computation. In \emph{Proceedings of the 25th IEEE International Conference on Distributed Computing Systems (ICDCS'05).}533-542.
    
\end{thebibliography}
\end{frame}

\end{document}



%\begin{block}{Examples}
%Some examples of commonly used commands and features are included, to help you get started.
%\end{block}



%\section{Some \LaTeX{} Examples}
%\subsection{Tables and Figures}
%\begin{frame}{Tables and Figures}

%\begin{itemize}
%\item Use \texttt{tabular} for basic tables --- see Table~\ref{tab:widgets}, for example.
%\item You can upload a figure (JPEG, PNG or PDF) using the files menu. 
%\item To include it in your document, use the \texttt{includegraphics} command (see the comment below in the source code).
%\end{itemize}

% Commands to include a figure:
%\begin{figure}
%\includegraphics[width=\textwidth]{your-figure's-file-name}
%\caption{\label{fig:your-figure}Caption goes here.}
%\end{figure}

%\begin{table}
%\centering
%\begin{tabular}{l|r}
%Item & Quantity \\\hline
%Widgets & 42 \\
%Gadgets & 13
%\end{tabular}
%\caption{\label{tab:widgets}An example table.}
%\end{table}
%\end{frame}

%\subsection{Mathematics}

%\begin{frame}{Readable Mathematics}

%Let $X_1, X_2, \ldots, X_n$ be a sequence of independent and identically distributed random variables with $\text{E}[X_i] = \mu$ and $\text{Var}[X_i] = \sigma^2 < \infty$, and let
%$$S_n = \frac{X_1 + X_2 + \cdots + X_n}{n}
%      = \frac{1}{n}\sum_{i}^{n} X_i$$
%denote their mean. Then as $n$ approaches infinity, the random variables $\sqrt{n}(S_n - \mu)$ converge in distribution to a normal $\mathcal{N}(0, \sigma^2)$.

%\end{frame}


%\section{Search Customer Data with SELRank}
%\begin{frame}{Search Customer Data with SELRank}

%There are two scenarios to search the data:
%\begin{enumerate}
%    \item Keyword Based Search
%    \item Product Based Query 
%\end{enumerate}

%\includegraphics[width=\textwidth]{images/Figure7.png}
%\end{frame}
